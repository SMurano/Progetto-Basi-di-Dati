\chapter{Modello Logico}

FOTO(\underline{CodFoto},Dispositivo,DimAltezza,DimLarghezza,\newline\uparrow{NomeLuogo},NomeFoto,TipoFoto, \uparrow{NFotografo})
\newline
FOTO.Nfotografo\leftarrow{UTENTE.Nickname}
\newline
FOTO.NomeLuogo\leftarrow{LUOGO.NomeLuogo}
\newline
\newline
LUOGO(\underline{NomeLuogo},Latitudine, Longitudine)
\newline
\newline
SOGGETTO(\underline{CodSoggetto}, Tipo, \uparrow{NomeSoggetto},\uparrow{NickSoggetto})
\newline
SOGGETTO.NomeSoggetto\leftarrow{LUOGO.NomeLuogo}
\newline
SOGGETTO.NickSoggetto\leftarrow{UTENTE.Nickname}
\newline
\newline
AFFERENZA(\underline{CodFoto, CodSoggetto})
\newline
AFFERENZA.CodFoto\leftarrow{FOTO.CodFoto}
\newline
AFFERENZA.CodSoggetto\leftarrow{SOGGETTO.CodSoggetto}
\newline
\newline
UTENTE(Nome, Cognome,\underline{Nickname},\uparrow{CODGP})
\newline
UTENTE.CodGP\leftarrow{GalleriaPersonale.CodGP}
\newline
\newline
GALLERIA PERSONALE(\underline{CODGP}, NomeGP)
\newline
\newline
GALLERIA CONDIVISA(\underline{CODGC}, NomeGC)
\newline
\newline
PARTECIPAZIONE(\underline{Nickname,CODGC})
\newline
PARTECIPAZIONE.Nickname\leftarrow{UTENTE.Nickname}
\newline
PARTECIPAZIONE.CODGC\leftarrow{GalleriaCondivisa.CODGC}
\newline
\newline
CONDIVISIONE(\underline{CodFoto,CODGC})
\newline
CONDIVISIONE.CodFoto\leftarrow{FOTO.CodFoto}
\newline
CONDIVISIONE.CODGC\leftarrow{GalleriaCondivisa.CODGC}
\newline
\newline
CONTENIMENTO(\underline{CodFoto, CodGP})
\newline
CONTENIMENTO.CodFoto\leftarrow{FOTO.CodFoto}
\newline
CONTENIMENTO.codGP\leftarrow{GalleriaPersonale.CodGP}
\newline

\section{Trigger e Procedure}
Nella seguente sezione vengono riportati tutti i Trigger e le procedure che sono utilizzate nel Data Base relazionale, riportate con il loro nome e la descrizione delle loro operazioni ed utilità.
\newline
\newline
\textbf{CreazioneGalleriaPersonale}: Trigger Function che associa ad ogni nuovo utente, una galleria personale con il nome di "Galleria di Nickname", dove Nickname è il Nickname dell'utente. Utile per avere un uniformità del nome delle gallerie personali.
\newline
\newline
\textbf{RinominaGalleriaCondivisa}:Trigger Function che concatena al nome della galleria condivisa, il corrispettivo codice preceduto dal carattere "Cancelletto".
\newline
\newline
\textbf{CreaElencoUtenti}:Trigger Function che riempie il campo
Elencoutenti della tabella Condivisione. Permette di
condividere una foto in una galleria senza fornirne l’elenco di
partecipanti.
\newline
\newline
\textbf{EliminaUtenteDaElenco}:Procedura che permette ad un
membro di una galleria condivisa di eliminare una foto per sè
lasciandola visibile per gli altri utenti.
\newline
\newline
\newline
\textbf{FiltraFotoPerLuogo}:Funzione che permette di visualizzare tutte
le foto scattate nello stesso luogo.
\newline
\newline
\textbf{FiltraFotoPerSoggetto}:Funzione che permette di visualizzare
tutte le foto che condividono lo stesso soggetto.
\newline
\newline
\textbf{Top3LuoghiPiuImmortalati}:Trigger Function che mantiene
aggiornata la View contenente i 3 luoghi più immortalati in
tutto il database.
\newline
\newline
\textbf{PrivatizzaFoto}:Trigger Function che rimuove una foto resa
privata da tutte le gallerie in cui è condivisa.
\newline
\newline
\textbf{EliminaFotoPrivataGC}:Trigger Function di controllo. Garantisce
che non ci siano foto private in gallerie condivise.
\newline
\newline
\newline
\textbf{EliminaFoto}:Trigger Function che rimuove una foto dal
database se non presente in nessuna galleria, sia personale che
condivisa.
\newline
\newline
\textbf{EliminaLuogo}:Trigger Function che rimuove un luogo dal
database se non associato a nessuna fotografia.
\newline
\newline
\textbf{EliminaSoggetto}:Trigger Function che rimuove un soggetto dal
database se non associato a nessuna fotografia.
\newline
L'utilizzo di tali funzioni (EliminaFoto,EliminaLuogo,ElimanaSoggetto), vengono chiamate ad ogni cancellazione di foto nel database e controllano che i dati relativi alla foto cancellata vengano
correttamente gestiti.
\newpage
\textbf{EliminaFotoLibera}:Trigger Function di controllo. Garantisce che
nel database non esistano foto che non sono contenute in
alcuna galleria (sia privata che condivisa).
\newline
\newline
\textbf{EliminaSoggettoLibero}:Trigger Function di controllo.
Garantisce che nel database non esistano soggetti che non
afferiscono ad alcuna foto.
\newline
\newline
\textbf{EliminaLuogoLibero}: Trigger Function di controllo. Garantisce
che nel database non esistano luoghi che non localizzano o
sono soggetti di alcuna foto.
\newline
\newline
\textbf{EliminaGalleriaVuota}:Trigger Function di controllo. Garantisce
che non ci siano gallerie condivise con meno di due partecipanti.
\newline
Le funzioni (EliminaFotoLibera, EliminaSoggettoLibero,
EliminaLuogoLibero, EliminaGalleriaVuota) vengono chiamate
periodicamente e controllano che tutte le foto, soggetti e
luoghi siano consistenti.