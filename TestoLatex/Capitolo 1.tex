\chapter{Analisi del Progetto e requisiti individuati}
\section{Obiettivi}
L’obiettivo del progetto descritto in tale documento è di realizzare un sistema informativo per la gestione di collezioni fotografiche geolocalizzate condivise.
\section{Requisisti sui dati}
Il sistema dovrà contenere informazioni relative:
\begin{itemize}
    \item Gli utenti del sistema
        \begin{itemize}
            \item Le credenziali di accesso.
             \item Informazioni Anagrafiche dell’utente
                \begin{itemize}
                    \item Nome dell’utente
                    \item Cognome dell’utente
                \end{itemize}
        \end{itemize}
    \item Le fotografie scattate dagli utenti del sistema
        \begin{itemize}
            \item L’utente Autore della fotografia
            \item la tipologia di dispositivo di scatto
            \item Il luogo in cui avviene lo scatto
        \end{itemize}
    \item I luoghi in cui avvengono scatti
        \begin{itemize}
            \item Coordinate geografiche del luogo
            \item Eventuale nome del luogo
        \end{itemize}
    \item I soggetti immortalati nelle fotografie
        \begin{itemize}
            \item Tipologia di soggetto immortalato
        \end{itemize}
\end{itemize}

\section{Requisiti funzionali}
Ogni utente avrà accesso ad una sua personale schermata Home in cui potrà:
\begin{itemize}
    \item Visualizzare la propria galleria personale
        \begin{itemize}
            \item Aggiungere la foto alla propria galleria personale.
            \item Eliminare foto dalla propria galleria personale.
            \item Filtrare le foto della propria galleria personale per luogo di scatto.
            \item Visualizzare una classifica dei Top 3 luoghi più immortalati dalle foto della galleria.
        \end{itemize}
    \item Partecipare a gallerie condivise con gli altri utenti
        \begin{itemize}
            \item Condividere fotografie personali con altri utenti, attraverso la galleria che hanno in comune.
            \item Rendere private le fotografie personali condivise con gli altri utenti, in
            modo tale da oscurare a tutti i membri della galleria la foto, che una volta resa privata, tornerà ad essere a disposizione per l’utente che ha scattato la foto nella sua galleria personale, ma non sarà presente in galleria condivisa.
            \item Eliminare fotografie dalla propria vista della galleria condivisa, in modo tale che la foto non sia più visibile all’utente che ha eliminato la foto, ma sia visibile a tutti gli altri membri che fanno parte della galleria condivisa.
            \item Filtrare le foto per soggetto immortalato della propria vista della galleria condivisa.
            \item Visualizzare una classifica dei Top 3 luoghi più immortalati delle foto della galleria.
        \end{itemize}
\end{itemize}

\section{Requisiti Aggiuntivi}
\begin{itemize}
    \item Utente: In relazione all’entità utente, nei suoi attributi è presente il Nome, il cognome e password di accesso alla propria area privata.
    \item FOTO: In relazione all’entità foto, nei suoi attributi è presente la data di scatto della foto, la larghezza e la lunghezza della foto.
\end{itemize}